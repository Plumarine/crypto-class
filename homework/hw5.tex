
\documentclass[12pt]{article}


\usepackage{amsmath}
\usepackage{amssymb}
\usepackage{attrib}
\usepackage{color}
\usepackage{fancyhdr}
\usepackage{graphicx}
\usepackage{hyperref}
\usepackage{makeidx}
\usepackage{stmaryrd}
\usepackage{xcolor}

\usepackage{preamble/lgarron-1.4.1}


\hypersetup{%
  colorlinks=true,% hyperlinks will be coloured
  linkcolor=blue,% hyperlink text will be green
  linkbordercolor=blue,% hyperlink border will be red
}

\def\rf{{\ \overset{R}\longleftarrow\ }}

\makeindex

\usepackage{multirow}

\usepackage{fancyhdr}
\pagestyle{fancy}
\fancyhead[C]{SPCS Cryptography -- Homework 5}
\rhead{}

\lhead{}

\let\sol=\undefined
%\newcommand{\sol}[1]{\begin{proof}\color{red}\textbf[SOLUTION: {#1}]\end{proof}}
\newcommand{\sol}[1]{}


%%%%%%%%%%%%%%%%%%%%%%%%%%%%%%%%%%%%%%%%%%%%%%%%%%%%%%%%%%%%%%%%
  
\begin{document} 

\section{SPCS}

\subsection{}

\section{Chinese Remainder Theorem}

Find numbers that satisfy the following congruences:

\subsection{}

\begin{itemize}
\item $x \equiv 2 ~(\bmod~5)$
\item $x \equiv 1 ~(\bmod~7)$
\end{itemize}

\sol{22}

\subsection{}

\begin{itemize}
\item $x \equiv 9 ~(\bmod~13)$
\item $x \equiv 4 ~(\bmod~24)$
\end{itemize}

\sol{100}

\subsection{}

\begin{itemize}
\item $x \equiv 1 ~(\bmod~2)$
\item $x \equiv 1 ~(\bmod~3)$
\item $x \equiv 2 ~(\bmod~5)$
\end{itemize}

\sol{7}

\subsection{}

\begin{itemize}
\item $x \equiv 4 ~(\bmod~7)$
\item $x \equiv 3 ~(\bmod~11)$
\item $x \equiv 8 ~(\bmod~13)$
\end{itemize}

\sol{333}

\section{Chinese Remainder Theorem with Friends}

Pick a partner you haven't worked with before.

Ask them to select a number $x$ from $1$ to $60$ and tell you:

\begin{itemize}
\item $x ~(\bmod~3)$
\item $x ~(\bmod~4)$
\item $x ~(\bmod~5)$
\end{itemize}

Now, calculate $x$ (if you get something outside of the range $1, ..., 60$, remember that you can add ay multiple of 60 and the congruences remain valid).


\section{Speeding up RSA}

Do an RSA encryption with a partner as in previous homeworks. However, when you decrypt, calculate:

\begin{itemize}
\item $c^d \bmod~p$
\item $c^d \bmod~q$
\end{itemize}

Now, use these values to calculate $c^d \bmod~N$ using the Chinese Remainder Theorem.

\section{Factoring}

\subsection{}

Factor the following numbers using trial division (you can use a calculator, but don't use a function that factors numbers for you):

\begin{itemize}
\item $34111$
\item $11121$
\item $1001$
\item $483$
\end{itemize}

Make sure every factor you end up with is prime.

\subsection{}


Use the difference of two squares to factor the following numbers:

\begin{itemize}
\item 323 \sol{(18+1)(18-1)}
\item 105 \sol{(13+8)(13-8)}
\item 221 \sol{(13+8)(13-8)}
\item 8099 \sol{(93+4)(93-4)}
\end{itemize}

\subsection{}

Use the Fermat primality test to find witnesses and prove numbers from the previous section composite.


\section{RSA Blind Signatures}

\subsection{}

Let's show that it's possible to sign a message using RSA without knowing what the message is.

Suppose Bob has a private key $(N, e)$ for the RSA signature scheme, and Alice wants him to sign $m$. Alice does the following:

\begin{itemize}
\item Select a random $r \in \Z_N^*$.
\item Calculate $r' = r^d$ This is the \emph{blinding value}.
\item Blind $m$ using $r'$ to by multiplying: $m' = m \* r'$.
\item Ask Bob to sign $m'$ to get $s' = Sign(sk, m')$.
\item (Convert $s'$ into a valid signature of $s$.)
\end{itemize}

Calculate the value of $s'$ in terms of  $m$, $r$, and the RSA parameters $(N, e, d)$. Use this to figure out sweat should go in the last step. That is, how Alice can use $s'$ to get a valid signature $s$ that will pass verification:

$$\text{Verify}(pk, s, m) = \Big( m \overset{?}{=} s^d \bmod~N \Big)$$

Explain why it is impossible for Bob to find out what message $m$ he was signing if he doesn't know $r$ or $r'$.

In particular, can you show that it is \emph{completely} impossible for Bob to get any information about $m$? (Except that $m \in \Z_n$, which he already knows.)


\subsection{}

Try this operation with a partner.

\subsection{}

Is this cool or what?


\end{document}
