\documentclass[12pt]{article}


\usepackage{amsmath}
\usepackage{amssymb}
\usepackage{attrib}
\usepackage{color}
\usepackage{fancyhdr}
\usepackage{graphicx}
\usepackage{hyperref}
\usepackage{makeidx}
\usepackage{stmaryrd}
\usepackage{xcolor}

\usepackage{preamble/lgarron-1.4.1}


\hypersetup{%
  colorlinks=true,% hyperlinks will be coloured
  linkcolor=blue,% hyperlink text will be green
  linkbordercolor=blue,% hyperlink border will be red
}

\def\rf{{\ \overset{R}\longleftarrow\ }}

\makeindex

\usepackage{multirow}

\usepackage{subfigure}
\usepackage{hhline}

\usepackage{fancyhdr}
\pagestyle{fancy}
\fancyhead[C]{SPCS Cryptography -- Homework 12}
\rhead{}

\lhead{}


\let\sol=\undefined
%\newcommand{\sol}[1]{\begin{proof}\color{red}\textbf[SOLUTION: {#1}]\end{proof}}
\newcommand{\sol}[1]{}


%%%%%%%%%%%%%%%%%%%%%%%%%%%%%%%%%%%%%%%%%%%%%%%%%%%%%%%%%%%%%%%%
  
\begin{document}

\section{Encryption}

Consider the following $PRP$, $E(k, x)$:

$$
E : \mathcal{K} \times \mathcal{X} \to \mathcal{X}
$$

where

$$
\mathcal{K} = \{\text{ant},\text{bird},\text{cat},\text{dog},\text{eel},\text{fish}\}
$$
$$
\mathcal{X} = \{0, 1\}^3
$$

$$
\large
\begin{array}{l||c|c|c|c|c|c|c|c|c}
E & 000 & 001 & 010 & 011 & 100 & 101 & 110 & 111 & x\\ \hhline{|=||=|=|=|=|=|=|=|=|=}
 \text{ant} & 000 & 100 & 110 & 001 & 111 & 010 & 011 & 101 \\ \hline
 \text{bird} & 111 & 001 & 000 & 110 & 100 & 010 & 101 & 011 \\ \hline
 \text{cat} & 011 & 100 & 001 & 010 & 101 & 000 & 110 & 111 \\ \hline
 \text{dog} & 000 & 100 & 110 & 010 & 101 & 011 & 001 & 111 \\ \hline
 \text{eel} & 110 & 100 & 101 & 011 & 001 & 111 & 010 & 000 \\ \hline
 \text{fish} & 000 & 001 & 100 & 110 & 011 & 101 & 010 & 111 \\ \hline
k
\end{array}
$$

\subsection{}

Fill in the inverse table for $D(k, x)$:

$$
\large
\begin{array}{l||c|c|c|c|c|c|c|c|c}
D & 000 & 001 & 010 & 011 & 100 & 101 & 110 & 111 & x\\ \hhline{|=||=|=|=|=|=|=|=|=|=}
 \text{ant} &  &  &  &  &  &  &  &  \\ \hline
 \text{bird} &  &  &  &  &  &  &  &  \\ \hline
 \text{cat} &  &  &  &  &  &  &  &  \\ \hline
 \text{dog} &  &  &  &  &  &  &  &  \\ \hline
 \text{eel} &  &  &  &  &  &  &  &  \\ \hline
 \text{fish} &  &  &  &  &  &  &  &  \\ \hline
k
\end{array}
$$

\newpage
\subsection{}

Based on your work, verify that this is a PRP. In particular:

\begin{itemize}
\item $f(x) = E(k, x)$ is a permutation for any specific key $k \in \mathcal{K}$.
\begin{itemize}
\item If you assign each value in $\{0, 1\}^3$ a distinct color and use this to color in the table, what would happen?
\end{itemize}
\item $f(x) = E(k, x)$ always has a unique inverse $f^{-1}(x) = D(k, x)$.
\begin{itemize}
\item This should hold because of the permutation property.
\end{itemize}
\item $E(k, x)$ and $D(k, x)$ are efficiently computable for any $k \in \mathcal{K}, x \in \mathcal{X}$.
\end{itemize}

Note that this is not a \emph{secure} PRP because $|\mathcal{K}|$ and $|\mathcal{X}|$ are very small. (The entire table fits on a single page!)

\subsection{}

Decrypt the following messages:

\begin{itemize}
\item Decrypt $(IV = {\tt 101}, c = {\tt 000~011~010 })$ using CBC with the key $\text{cat}$.
\begin{itemize}
\item (You should get ${\tt 000~000~000}$).
\end{itemize}
\item Decrypt $(IV = {\tt 110}, c = {\tt 010~111~111})$ using CTR with the key $\text{cat}$. \sol{${\tt 100~000~100}$} 
\end{itemize}

\section{}

Verify the following signatures:

\begin{itemize}
\item Verify $s = {\tt 110}$ for the message ${\tt 000}$ using CBC-MAC with the key $(k_1, k_2) = (\text{dog}, \text{eel})$.
\item Verify $s = {\tt 010}$ for the message ${\tt 001~010~011}$ using CBC-MAC with the key $(k_1, k_2) = (\text{dog}, \text{eel})$.
\end{itemize}

\newpage
For the following section, we will be using the authenticated encryption systems built from our and existing cipher $Encrypt, Decrypt$, and CBC-MAC. The encryption method is:\\

$Encrypt-Then-Sign\Big(\large(k_E, (k_1, k_2)\large), x\Big):$

\begin{itemize}
\item $(IV, c) = Encrypt(k_E, x)$
\item $s = Sign\Large((k_1, k_2), IV \| c\Large)$
\item output $(IV, c, s)$
\end{itemize}

\section{}

Why is it safe to concatenate the IV with c for the signature?

(Recall that excluding the IV completely allows for an easy forgery. Why can this not happen here?)

\section{}

Decrypt and verify the following authenticated encryption:

\begin{itemize}
\item $(IV_{CBC} = {\tt 111}, c = {\tt 111~111~000~001~111~100}, s = {\tt 001})$ using CBC with CBC-MAC:
\begin{itemize}
\item $k_{CBC} = \text{eel}$
\item $k_{CBC{\textunderscore}MAC} = (k_1, k_2) = (\text{cat}, \text{dog})$ \\\sol{{\tt 011}}
\end{itemize}
\end{itemize}


\subsection{}

Bonus: Unpad and decode the message using ASCII character codes. (Assume our standard padding scheme was used: add a one bit, then add zeros.)

\sol{
$${\tt 010~010~000~100~100~110}$$

ASCII version: ``HI''
}



\subsection{}

Decrypt and verify the following authenticated encryption:

\begin{itemize}
\item $(s = {\tt 011}, IV_{CTR} = {\tt 110}, c = {\tt 111~001~101~100~010~100})$ using CTR with CBC-MAC:
\begin{itemize}
\item $k_{CTR} = \text{bird}$
\item $k_{CBC{\textunderscore}MAC} = (k_1, k_2) = (\text{ant}, \text{fish})$ \\\sol{{\tt 011}}
\end{itemize}
\end{itemize}


\sol{
$${\tt 010~010~000~100~100~110}$$

ASCII: ``IT''
}

\subsection{}

Pick  partner you haven't worked with before.

Agree on a CTR key and a CBC-MAC (double) key with your partner. Ask them to send you \emph{two} authenticated encryptions of different $9$-bit messages:

\begin{itemize}
\item One with a valid signature.
\item One where they have either tampered with the encrypted ciphertext, the IV, or the MAC.
\end{itemize}

Check that exactly of the messages verifies, and decrypt the valid one.

(If you want, use a longer ASCII-encoded message instead of 9 bits. Make sure you both know what padding you're using.)

\subsection{}

Pick one of the ways CBC-MAC can be broken:
\begin{itemize}
\item ``Randomized'' $IV$
\item No finalization (no extra encryption at the end).
\item Same key for the CBC portion and the the finalization.
\end{itemize}

Ask Lucas/Aaron/Conrad queries in the MAC Security game and produce a forgery.


\newpage
\section{One-Time MAC}

Recall our one-time MAC from class:


\begin{itemize}
\item $p \in \mathbb{P}$
\item $(a, b) \rf \Z_p^* \times \Z_p$
\item $Sign\Big((a, b), x\Big) = a\*x + b \mod p$
\end{itemize}
%
%\subsection{}
%
%The signature scheme as written doesn't quite work. It should be:
%
%\begin{itemize}
%\item $(a, b) \in \mathcal{A}, \mathcal{B}$
%\item $Sign((a, b), x) = \mathcal{S}$, where $x \in \mathcal{X}$
%\end{itemize}
%
%In order for correctness and security to hold, what should $\mathcal{A}, \mathcal{B}, \mathcal{S}, \mathcal{X}$ each be?
%
%(Choose from $\Z_p$ or $\Z_p^*$ for each one. Remember that $p$ will be a large prime in a secure version of this system.)

\subsection{}

Say we have a particular key $(a, b)$.

Prove that each different message would result in a different message if we sign it (using this particular key).

\subsection{}

If $p$ is large enough and $(a, b)$ is a completely random valid key, can you use this as a secure PRP?

\sol{

No, two queries and you've broken it.

}


\subsection{}

Let $(a, b) = (3, 4) \in \Z_7^* \times \Z_7$ (i.e. $p = 7$).

\begin{itemize}
\item Compute the signatures of $x = 0, 1, ..., 6$
\end{itemize}

\subsection{}

Here are two valid message-signature pairs computed using the same key with $p=13$

\begin{itemize}
\item $(x=4, s=5)$
\item $(x=2, s=4)$
\end{itemize}

Produce a valid forgery for the message $x = 10$.

(Can you find an easy way to do it in this case, without solving for $a$ and $b$?)

\subsection{}

Do the previous problem with a partner, using $p = 17$

\begin{itemize}
\item Ask them to select a random key $(a, b) \rf \Z_{17}^* \times \Z_{17}$.
\item Send them two messages $x$ and $x'$ and receive a signature on each.
\item Produce a forgery on a new message $x''$.
\end{itemize}

\subsection{}

Open-Ended: Can you fix this system so that it allows more than two uses?




\end{document}
