
\documentclass[12pt]{article}


\usepackage{amsmath}
\usepackage{amssymb}
\usepackage{attrib}
\usepackage{color}
\usepackage{fancyhdr}
\usepackage{graphicx}
\usepackage{hyperref}
\usepackage{makeidx}
\usepackage{stmaryrd}
\usepackage{xcolor}

\usepackage{preamble/lgarron-1.4.1}


\hypersetup{%
  colorlinks=true,% hyperlinks will be coloured
  linkcolor=blue,% hyperlink text will be green
  linkbordercolor=blue,% hyperlink border will be red
}

\def\rf{{\ \overset{R}\longleftarrow\ }}

\makeindex

\usepackage{multirow}

\usepackage{fancyhdr}
\pagestyle{fancy}
\fancyhead[C]{SPCS Cryptography -- Homework 4}
\rhead{}

\lhead{}

\let\sol=\undefined
\newcommand{\sol}[1]{\begin{proof}\color{red}\textbf[SOLUTION: {#1}]\end{proof}}
%\newcommand{\sol}[1]{}


%%%%%%%%%%%%%%%%%%%%%%%%%%%%%%%%%%%%%%%%%%%%%%%%%%%%%%%%%%%%%%%%
  
\begin{document} 


\section{ElGamal Encryption}

Recall ElGamal:


\subsubsection*{$KeyGen()$}


\begin{itemize}
\item $p \in \mathbb{P}$
\item $g \in \Z_p$ is a generator
\item $x \rf \{1, ..., p-1\}$
\item $h = g^x$
\item $pk$ $= (p, g, h)$
\item $sk$ $= (x)$
\end{itemize}

\subsubsection*{$E(pk, m)$}


\begin{itemize}
\item $y \rf \{1, ..., p-1\}$
\item Send the ciphertext $c = (g^y, m \* h^y)$
\end{itemize}

\subsubsection*{$D(sk, c)$}


\begin{itemize}
\item Label the parts of the ciphertext as $c = (c_1, c_2)$
\item Decrypt the message as: $c_2 \* ({c_1}^x)^{-1}$
\end{itemize}


\subsection{}


Consider the following ElGamal setup:

\begin{itemize}
\item $pk = (G = \Z_{31}^*, g = 3, h = 6)$
\item $sk = (x = 25)$
\end{itemize}

Decrypt the following cipher texts:

\begin{itemize}
\item $(18, 19)$ \sol{10}
\item $(22, 23)$ \sol{14}
\item $(19, 2)$ \sol{10}
\end{itemize}

\subsection{}


Ask a partner to select a message $m \in \Z_{31}^*$ and encrypt it for you. Decrypt it.

\subsection{}

How many possible ciphertext pairs $(c_1, c_2)$ are there for a given $p$ and a specific value of $x$. How many valid values of $x$ are there?


\sol{
\begin{itemize}
\item $(p-1)^2$
\item $(p-1)$
\end{itemize}


}

\section{RSA without $p$ or $q$}

\subsection{}


\label{rsa_p_q}

Suppose you have a private RSA key with $p$ and $q$ missing, but you have $\varphi(N)$. That is, you have:

$$\Big(N, e, \varphi(N)\Big)$$

Show that you can still find $d$ and decrypt messages encrypted as $m^e ~(\bmod~N)$.

\sol{
$$d = e^{-1} \bmod~N$$

You can find $e$ using the Extended Euclidean Algorithm.
}

\subsection{}

Apply your solution to $\Big(N = 91,~e = 5, varphi(N) = 72\Big)$

\section{Reusing RSA Primes}

\label{rsa_reuse}

In class, we showed that it's safe for everyone to use the same group and generator. Let's show that sharing the same modulus $N$ doesn't work for RSA.

Suppose we try following:

\begin{itemize}
\item Generate $N = p \* q$ as in RSA.
\item Generate \emph{two} pairs of exponents (just do this step from normal RSA two separate times): $(e_a, d_a)$ and $(e_b, d_b)$
\item Publish $pk_a = (N, e_a)$ and give $sk_a = (N, e_a, d_a)$ to Alice.
\item Publish $pk_b = (N, e_b)$ and give $sk_b = (N, e_b, d_b)$ to Bob.
\end{itemize}

Our goal is that anyone can encrypt to either Alice or Bob (using their corresponding public key), but neither Alice nor Bob can read messages intended for the other.

\begin{itemize}
\item Why can't we give $p$ or $q$ to Alice (or Bob)?
\item Verify that Alice and Bob can still decrypt messages intended for them.
\end{itemize}

Now, show that the system is broken: Alice can actually decrypt any message for Bob.\\

(Hint: First, show that Alice can use her private key to find a number that is $\equiv 1 \mod \varphi(N)$. Then use problem \ref{rsa_p_q} to show that she can find a decryption exponent $d_b'$ that allows her to decrypt messages by Bob using normal RSA decryption. Is it necessarily true that $d_b = d_b'$?)

\sol{
Use the Extended Euclidean Algorithm to find $e \* x + k \* \varphi(N) \* y = 1$.

}


\section{Even more Hardcore RSA}

Show that in problem \ref{rsa_reuse}, Alice can actually factor $N$ to recover $p$ and $q$.

(This one is hard.)

\sol{See page 3 of \href{https://crypto.stanford.edu/~dabo/papers/RSA-survey.pdf}{https://crypto.stanford.edu/~dabo/papers/RSA-survey.pdf} (``Twenty Years of Attacks on the RSA Cryptosystem'')}


\section{RSA Blinding}

\subsection{}

Let's show that it's possible to sign a message using RSA without knowing what the message is.

Suppose Bob has a private key $(N, e)$ and Alice wants him to sign $m$. She can do the following:

\begin{itemize}
\item Select a random $r \in \Z_N^*$.
\item Calculate $r' = r^d$ This is the \emph{blinding value}.
\item Ask Bob to sign $m \* r'$ to get $s = Sign(sk, m \* r')$.
\item Perform a multiplication on $s$ to get a valid signature of $E(sk, m)$.
\end{itemize}

Find what they need to multiply in the last step to make this work, and argue why the whole procedure prevents you from knowing anything about $m$.\\

Why do you think this is called ``blinding''?

\sol{
$s$
}

\subsection{}

Consider the following procedure.

\begin{itemize}
\item Select a random $r \in \Z_N^*$. This is the \emph{blinding value}
\item Perform a multiplication on $m$ (probably using $r$) to get a value $m'$.
\item Ask you to sign $m'$. to get $s = E(sk, m')$
\end{itemize}

Can you find a multiplication for the second step so that $s$ is a valid signature for $m$ without any additional calculations for them?


\section{ElGamal Rerandomization}

\label{rerandomize}

Recall that an ElGamal encryption is of the form

$$
(g^y, m \* h^y)
$$

where $x \rf Exponents$ and $h = g^x$.

Take $z$ to be any valid exponent. Show that it is possible for someone (who doesn't know $x$) to modify this so that the result $(c_1, c_2)$ is the same as if the encryptor used $y' = y * z$ instead of $y$. That is, transform the original ciphertext into a valid ElGamal ciphertext of the format:

$$
(g^{y'}, m \* h^{y'})
$$

Verify that this decrypts.

\section{ElGamal: Additively Homomorphic Encryption}

Let $\mathcal{E}(m)$ be an algorithm that encrypts $m$ to a specific person (i.e. choose values for $G, g$, and $h$ and use the same ones every time). If you run it twice on the different messages $m_1$, $m_2$, you get:

\begin{itemize}
\item $a_1 = (g^{r_1}, m_1 \* h^{r_1})$
\item $a_2 = (g^{r_2}, m_1 \* h^{r_2})$
\end{itemize}

Come up with an algorithm $\mathcal{E'}(a_1, a_2)$ that produces a valid encryption of the message $m_1 + m_2$


Note that if you were doing this in practice, you should make sure to rerandomize (see problem \ref{rerandomize}).


%Why can't ElGamal be used as sig?
%ElGamal what is it missing compared to Cramer-Shoup?


\section{Programming}

If you know how to program, implement one of these: 

\begin{itemize}
\item Diffie-Hellman
\item RSA
\item ElGamal.
\end{itemize}


Talk to Conrad, Aaron, or Lucas if you'd like permission to work on this at the computers during study time.


\newpage

(Do problems 1-8 first. We'll talk about the Chinese Remainder Theorem during study time if you finish all the other problems. :-P)

\section{Chinese Remainder Theorem}

Find numbers that satisfy the following congruences:

\subsection{}

\begin{itemize}
\item $x \equiv 2 ~(\bmod~5)$
\item $x \equiv 1 ~(\bmod~7)$
\end{itemize}

\sol{22}

\subsection{}

\begin{itemize}
\item $x \equiv 9 ~(\bmod~13)$
\item $x \equiv 4 ~(\bmod~24)$
\end{itemize}

\sol{100}

\subsection{}

\begin{itemize}
\item $x \equiv 1 ~(\bmod~2)$
\item $x \equiv 1 ~(\bmod~3)$
\item $x \equiv 2 ~(\bmod~5)$
\end{itemize}

\sol{7}

\subsection{}

\begin{itemize}
\item $x \equiv 4 ~(\bmod~7)$
\item $x \equiv 3 ~(\bmod~11)$
\item $x \equiv 8 ~(\bmod~13)$
\end{itemize}

\sol{333}

\section{Chinese Remainder Theorem with Friends}

Pick a partner you haven't worked with before.

Ask them to select a number $x$ from $1$ to $60$ and tell you:

\begin{itemize}
\item $x ~(\bmod~3)$
\item $x ~(\bmod~4)$
\item $x ~(\bmod~5)$
\end{itemize}

Now, calculate $x$ (if you get something outside of the range $1, ..., 60$, remember that you can add ay multiple of 60 and the congruences remain valid).


\section{Speeding up RSA}

Do an RSA encryption with a partner as in yesterday's homework. However, when you decrypt, calculate:

\begin{itemize}
\item $c^m \bmod~p$
\item $c^m \bmod~q$
\end{itemize}

Now, use these values to calculate $c^m \bmod~N$ using the Chinese Remainder Theorem.







\end{document}
