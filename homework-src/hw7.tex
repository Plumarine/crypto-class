
\documentclass[12pt]{article}


\usepackage{amsmath}
\usepackage{amssymb}
\usepackage{attrib}
\usepackage{color}
\usepackage{fancyhdr}
\usepackage{graphicx}
\usepackage{hyperref}
\usepackage{makeidx}
\usepackage{stmaryrd}
\usepackage{xcolor}

\usepackage{preamble/lgarron-1.4.1}


\hypersetup{%
  colorlinks=true,% hyperlinks will be coloured
  linkcolor=blue,% hyperlink text will be green
  linkbordercolor=blue,% hyperlink border will be red
}

\def\rf{{\ \overset{R}\longleftarrow\ }}

\makeindex

\usepackage{multirow}

\usepackage{fancyhdr}
\pagestyle{fancy}
\fancyhead[C]{SPCS Cryptography -- Homework 7}
\rhead{}

\lhead{}

\let\sol=\undefined
\newcommand{\sol}[1]{\begin{proof}\color{red}\textbf[SOLUTION: {#1}]\end{proof}}
%\newcommand{\sol}[1]{}


%%%%%%%%%%%%%%%%%%%%%%%%%%%%%%%%%%%%%%%%%%%%%%%%%%%%%%%%%%%%%%%%
  
\begin{document}

\section{Don't Reuse that One-Time Pad}

Here are five $4$-letter (upperase) messages in ASCII each XORed using the same key $k \in \{0, 1\}^{32}$. Recover all the messages and then encrypt the word MATH using the same key:

\begin{itemize}
\item $E(k, m_1) = \tt{10011111~01110001~01100001~11001101}$
\item $E(k, m_2) = \tt{10001110~01100010~01110111~11001011}$
\item $E(k, m_3) = \tt{10000000~01110001~01101011~11001010}$
\item $E(k, m_4) = \tt{10001110~01100010~01110111~11010111}$
\item $E(k, m_5) = \tt{10000110~01110001~01110111~11001101}$
\end{itemize}


Here's a table of ASCII values for the uppercase letters:

\begin{center}
\begin{tabular}{|c||c|c|c|c||c|c|}
\hline
A & 65 & \tt{01000001} & & N & 78 & \tt{01001110} \\
B & 66 & \tt{01000010} & & O & 79 & \tt{01001111} \\
C & 67 & \tt{01000011} & & P & 80 & \tt{01010000} \\
D & 68 & \tt{01000100} & & Q & 81 & \tt{01010001} \\
E & 69 & \tt{01000101} & & R & 82 & \tt{01010010} \\
F & 70 & \tt{01000110} & & S & 83 & \tt{01010011} \\
G & 71 & \tt{01000111} & & T & 84 & \tt{01010100} \\
H & 72 & \tt{01001000} & & U & 85 & \tt{01010101} \\
I & 73 & \tt{01001001} & & V & 86 & \tt{01010110} \\
J & 74 & \tt{01001010} & & W & 87 & \tt{01010111} \\
K & 75 & \tt{01001011} & & X & 88 & \tt{01011000} \\
L & 76 & \tt{01001100} & & Y & 89 & \tt{01011001} \\
M & 77 & \tt{01001101} & & Z & 90 & \tt{01011010} \\
\hline
\end{tabular}
\end{center}

\vspace{2em}

\sol{
Hint: E is the most common letter of the alphabet.


Key $k$: 11001011 00110100 00110010 10011001

Words: TEST, EVER, KEYS, EVEN, MEET

$$E(k, ``MATH'') = \tt{10000110~01110101~01100110~11010001}$$

}




\section{One-Time Pad Overkill}

\subsection{}


Suppose we use a double-key for OTP:

$$\mathcal{K} = \mathcal{K}_1 \times \mathcal{K}_2 = \{0, 1\}^n \times \{0, 1\}^n$$
$$\mathcal{M} = \{0, 1\}^n$$
$$\mathcal{C} = \{0, 1\}^n$$

\begin{itemize}
\item $E((k_1, k_2), m)  = k_1 \oplus k_2 \oplus m$
\item $D((k_1, k_2), c)  = k_1 \oplus k_2 \oplus c$
\end{itemize}

Prove that this cipher has perfect secrecy. (Make sure to use the definition of perfect secrecy.
)

\sol{
For any $m \in \mathcal{M}, c \in \mathcal{C}$:
$$
Pr[E(k \rf \mathcal{K}, m) = c] = {2^n \over 2^n \* 2^n}
$$

Thus, the definition of perfect secrecy holds if we compare any two messages.
}



\subsection{}

Do the same for:

\begin{itemize}
\item $\mathcal{K} = \{0, 1\}^n \times \{0, 1\}$
\item $E((k_1, b), m)  = k_1 \oplus \underset{n}{\underbrace{bb...b}} \oplus m$
\end{itemize}

(The bit string $\underset{n}{\underbrace{bb...b}}$ is $n$ copies of $b$.)

\sol{
For any $m \in \mathcal{M}, c \in \mathcal{C}$:
$$
Pr[E(k \rf \mathcal{K}, m) = c] = {2 \over 2^n \* 2}
$$
}




\section{Caesar Cipher's Shaky Secrecy (Say \emph{that} ten times.)}

Consider the Caesar Cipher:


$$\mathcal{M} = \mathcal{C} = \{0, ..., 25\}^n$$
$$\mathcal{K} = \{0, ..., 25\}$$

\begin{itemize}
\item $E(k, m) = \Big[(m[0] + k \bmod~26), ..., (m[n-1] + k \bmod~26)\Big]$
\item $D(k, c) = \Big[(c[0] - k \bmod~26), ..., (c[n-1] + k \bmod~26)\Big]$
\end{itemize}

Prove that it does not have perfect secrecy.


\sol{
Consider $m_0 = [0, 0]$ and $m_1 = [0, 1]$ and $c = [0, 0]$


Now:

$$Pr[E(k \rf \mathcal{K}, m_0) = c] = 1/26$$
$$Pr[E(k \rf \mathcal{K}, m) = c] = 0$$

Thus, the Caesar Cipher does not have perfect secrecy.
}

\section{Analyze that Cipher!\texttrademark}
Suppose we have a cipher

\begin{itemize}
\item $E : \mathcal{K} \times \mathcal{M} \to \mathcal{C}$
\item $D : \mathcal{K} \times \mathcal{C} \to \mathcal{M}$
\end{itemize}

that has perfect secrecy. Consider the following ciphers that are built on it. For each one:

\begin{itemize}
\item Verify that correctness holds (show that encrypting and decrypting a message $m$ using the same key always gives you back $m$).
\item Figure out if it has perfect secrecy. Prove/disprove it.
\end{itemize}


\subsection{}


\begin{itemize}
\item $\mathcal{K'} = \mathcal{K}$
\item $\mathcal{M'} = \mathcal{M}$
\item $\mathcal{C'} = \mathcal{C} \times \{0, 1\}^{8}$
\item $E'(k, m): = (E(k, m), \tt{01101100})$
\item $D'(k, (c_1, c_2))  = D(k, c_1)$
\end{itemize}

\sol{
Consider $c \in \mathcal{C}$. Then:

\begin{eqnarray*}
Pr[E(k, m) = c] & = & Pr[(E(k, m), \tt{01101100}) = (c, \tt{01101100})] \\
& = & Pr[E'(k, m) = (c, \tt{01101100})]
\end{eqnarray*}

Thus, we have perfect secrecy.

}

\subsection{}

\begin{itemize}
\item $\mathcal{K'} = \mathcal{K}$
\item $\mathcal{M'} = \mathcal{M}$
\item $\mathcal{C'} = \mathcal{C} \times \mathcal{K}$
\item $E'(k, m) = (E(k, m), k)$
\item $D'(k, (c_1, c_2) = D(k, c_1)$
\end{itemize}

\sol{
This does not have perfect secrecy.

Consider an encryption $c = (c_1, c_2) = E(k \rf \mathcal{K}, m)$ of a message $m$. This means that $k = c_2$.

Suppose another message $m'$  can encrypt to the same ciphertext. Since
$$E(k', m') = c = (c_1, c_2)$$
this means that $k' = c_2 = k$.

However, we also know that $D(k, c) = m$. Thus, two messages cannot encrypt to the same ciphertext, and we definitely don't have perfect secrecy.


}

\subsection{}

\begin{itemize}
\item $\mathcal{K'} = \mathcal{K} \times \mathcal{K}$
\item $\mathcal{M'} = \mathcal{M}$
\item $\mathcal{C'} = \mathcal{C} \times \mathcal{K}$
\item $E'((k_1, k_2), m) = (E(k_2 \oplus k_1, m), k_1)$
\item $D'((k_1, k_2), (c_1, c_2)) = D(k_2 \oplus k_1, c_1)$
\end{itemize}


\section{Bonus binary boolean function of the day: NOR}

\subsection{}


Consider the $NOR$ binary function

\begin{center}
\begin{tabular}{|c|c||c|}
\hline
$a$ & $b$ & $NOR(a, b)$ \\
\hline
0 & 0 & 1 \\
0 & 1 & 0 \\
1 & 0 & 0 \\
1 & 1 & 0 \\
\hline
\end{tabular}
\end{center}

Show how to write $a \oplus b$ using only (nested) $NOR$ functions with $a$ and $b$.

\sol{

$$NOR(NOR(a, b), NOR(NOR(a, a), NOR(b, b)))$$

}

\subsection{}


Try the same for $NAND$:

\begin{center}
\begin{tabular}{|c|c||c|}
\hline
$a$ & $b$ & $NAND(a, b)$ \\
\hline
0 & 0 & 1 \\
0 & 1 & 1 \\
1 & 0 & 1 \\
1 & 1 & 0 \\
\hline
\end{tabular}
\end{center}

\sol{

$$NAND(NAND(a, NAND(a, b)), NAND(b, NAND(a, b)))$$

}

\section{More RSA}

Suppose someone knows $\varphi(N)$ for a public RSA modulus $N$. Show how they can factor $N$ without additional information.

\sol{
\begin{itemize}
\item $N = p \* q$
\item $\varphi(N) = (p - 1) \* (q-1)$
\item $N = 1 - \varphi(N) = p + q$
\end{itemize}

We know $p \* q$ and $p + q$, so we can solve for $p$ and $q$ using standard quadratic techniques:

$$
p = \frac{1}{2} \left(-\sqrt{(\text{N}-\varphi(N) +1)^2-4
   \text{N}}+\text{N}-\varphi(N) +1\right)$$
$$q = \frac{1}{2}
   \left(\sqrt{(\text{N}-\varphi(N) +1)^2-4 \text{N}}+\text{N}-\varphi(N)
   +1\right)
$$


}

\section{More ElGamal Encryption}

Suppose Eve listens in on an ElGamal encryption from Alice to Bob. That is, she sees Bob publishing

$$(p, g, h = g^x)$$

and sees Alice sending

$$(c_1 = g^y, c_2 = m \* h^y)$$


Show that if Eve can find out $m$ without any additional information, she has broken an instance of the Computational Diffie-Hellman Problem (show the exact givens for the problem instance, and how she finds the secret value in the CDH Problem).


\sol{
Suppose Alice has $m$.

The instance is
$$
(g, g^x, g^y)
$$
and Eve finds
$$g^{xy}$$
by calculating
$$
c_2 \* m^{-1} (\bmod~p) = h^y = g^{xy}
$$
}




\end{document}
