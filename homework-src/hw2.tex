
\documentclass[12pt]{article}


\usepackage{amsmath}
\usepackage{amssymb}
\usepackage{attrib}
\usepackage{color}
\usepackage{fancyhdr}
\usepackage{graphicx}
\usepackage{hyperref}
\usepackage{makeidx}
\usepackage{stmaryrd}
\usepackage{xcolor}

\usepackage{preamble/lgarron-1.4.1}


\hypersetup{%
  colorlinks=true,% hyperlinks will be coloured
  linkcolor=blue,% hyperlink text will be green
  linkbordercolor=blue,% hyperlink border will be red
}

\def\rf{{\ \overset{R}\longleftarrow\ }}

\makeindex


\usepackage{fancyhdr}
\pagestyle{fancy}
\fancyhead[C]{SPCS Cryptography -- Homework 1}
\rhead{}

\lhead{}

\let\sol=\undefined
%\newcommand{\sol}[1]{\color{red}\textbf[SOLUTION: {#1}]}
\newcommand{\sol}[1]{}


%%%%%%%%%%%%%%%%%%%%%%%%%%%%%%%%%%%%%%%%%%%%%%%%%%%%%%%%%%%%%%%%
  
\begin{document} 

If you want to click on any of the links, you can find this at \href{https://garron.net/spcs/}{garron.net/spcs}

Today's homework is less math and a bit more exploring. You don't have to do everything. Make sure to do the math, but feel free to try the rest in any order.

\section{Extended GCD}

Calculate the extended GCD of the following:

\begin{itemize}
\item $GCD(4,30)$
\item $GCD(85,136)$
\item $GCD(4,7)$
\item $GCD(104,144)$
\item $GCD(34,55)$
\end{itemize}

\section{Inverses}

Calculate the following:

\begin{itemize}
\item $5^{-1} \mod 17$
\item $23^{-1} \mod 100$
\end{itemize}

Recall that $x^{-1} (mod n)$ is the number $y$ that makes $1 = x \* y (mod n)$.


\section{GCD Order}

If $x, y, z \in \Z^+$, prove that 

$$
GCD\Big(GCD(x, y), z\Big) = GCD\Big(x, GCD(y, z)\Big)
$$

(Hint: Try to extend the idea we used in class. Define $c$ as the ``the GCD of all three numbers'': the largest number in $\Z^+$ that divides all of $x, y$, and $z$. Then show that both sides must have the value $c$.)

\sol{Let $c \in \Z+$ be the largest number such that

$$
c | x\text{ and }c | y\text{ and }c | z
$$

}

\section{Your Friend, the Golden Ratio}

Do this in a large group:

Create a 25x25 grid and labels the rows and columns from $1$ to $25$ each. In the cell at row $x$ and column $y$, write down \emph{how many steps it takes for the Euclidean algorithm to finish} when calculating $GCD(x, y)$ (use the faster version, and don't count the first step if the numbers don't change). You should be able to find tricks for speeding this up because you're calculating ``in bulk''.

Circle the cells that have a larger value than anything ``before'' them. (i.e.where the value at $(x,y)$ is the largest for any $(x', y')$ with $x' \leq x, y\ \leq y$). Do you see a pattern? Can you explain it?


\section{Identity}


Prove that if $a, b \in \Z^+$, $a > b$, $GCD(a, b) = 1$, and $0 \leq m < n$, then:

$$
gcd(a^m - b^m, a^n - b^n) = a^{GCD(m, n)} - b^{GCD(m, n)}
$$

\sol{
This one's from \emph{Concrete Mathematics}, problem 4.38

}

\section{Thought Experiment}

Suppose Alice and Bob are trying to start a conversation that has confidentiality, authentication, and (mutual) identification. Assuming they take the following two steps (in either order), which of these do they need to do first, and why?

\begin{itemize}
\item Prove their identities to each other (Identification)
\item Agree on a shared secret key and use it to start an encrypted conversation (Confidentiality).
\end{itemize}

For now, assume they have a way to do both parts in an authenticated way (i.e. no one can tamper with what they send -- the receiver gets the ``authentic'' message that was sent from the other person, regardless of whether they trust the person or not).

\section{Repeated Squaring}

Calculate the following by repeated squaring:

\begin{itemize}
\item $4^{13} \mod 19$
\item $3^{20} \mod 17$
\end{itemize}


\section{Diffie-Hellman}

Pick a partner and perform Diffie-Hellman with:

\begin{itemize}
\item $G = \Z_{31}^*$
\item $g = 3$
\end{itemize}

That is, pick a random number $a \in \{1, ..., 30\}$ and send your partner $3^{a}$.

When you get a number from them (call it $h$), compute $h^a$. Compare your secrets.

(You'll probably need to do more repeated squaring.)

\section{Discrete Log}

Find $y$ such that $3^{y} \mod 17 = 12$

\section{Multi-Party Key Agreement}

Use Diffie-Hellman to design a protocol that allows 3 people to agree on a secret key.

Assume that all three can communicate to each other. Anyone can eavesdrop on the communication, but not modify it.

\begin{itemize}
\item Easy version: Once you have a secret key in common with anyone, you can communicate securely with them.
\item Easy version: You have to set up all the keys before you can communicate securely.
\end{itemize}


Try to demonstrate that breaking your system would break discrete log or Diffie-Hellman.




\end{document}
