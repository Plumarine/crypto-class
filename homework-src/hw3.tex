
\documentclass[12pt]{article}


\usepackage{amsmath}
\usepackage{amssymb}
\usepackage{attrib}
\usepackage{color}
\usepackage{fancyhdr}
\usepackage{graphicx}
\usepackage{hyperref}
\usepackage{makeidx}
\usepackage{stmaryrd}
\usepackage{xcolor}

\usepackage{preamble/lgarron-1.4.1}


\hypersetup{%
  colorlinks=true,% hyperlinks will be coloured
  linkcolor=blue,% hyperlink text will be green
  linkbordercolor=blue,% hyperlink border will be red
}

\def\rf{{\ \overset{R}\longleftarrow\ }}

\makeindex

\usepackage{multirow}

\usepackage{fancyhdr}
\pagestyle{fancy}
\fancyhead[C]{SPCS Cryptography -- Homework 3}
\rhead{}

\lhead{}


%%%%%%%%%%%%%%%%%%%%%%%%%%%%%%%%%%%%%%%%%%%%%%%%%%%%%%%%%%%%%%%%
  
\begin{document} 
\section{Homomorphisms}

Which of these are homomorphisms?

Either explain why, or give a counter-example to disprove it.

\begin{center}
\begin{tabular}{|l|c|c|c|}
\hline
& $f(x)$ & $a \star_1 b$ & $a \star_2 b$ \\
\hline
$f_1 : \R \to \R$ & $-x$ & $a + b$ & $a + b$ \\
\hline
$f_2 : \Z \to \Z$ & $x$ & $4b(a + b) + a^2$ & $(a + 2b)^2$ \\
\hline
$f_3 : \mathbb{P} \to \mathbb{P}$ & $1$ & $a$ & $a$ \\
\hline
$f_4 : \Z^+ \to \R$ & $x / 2$ & $a + b$ & $a + b$ \\
\hline
$f_5 : \Z^+ \to \R$ & $x / 2$ & $a \* b$ & $a \* b$ \\
\hline
\end{tabular}
\end{center}


\subsection{General}

$f : S \to S$ where:
\label{inv_hom}
\begin{itemize}
\item $S$ is the set of elements of \emph{any} group $G = (S, \star)$
\item $f(x) = x^{-1}$ (i.e. $f(x) =$ the inverse of $x$ in $G$)
\item $a \star_1 b = a \star b$
\item $a \star_2 b = b \star a$
\end{itemize}

\section{Semi-Generator}

Suppose $g$ is a generator of a group $G$. (Recall: This means that the powers of $g$ include every element in $G$.)

Show that $g^{-1}$ is also a generator.\\

(Hint: To convince yourself, try $G = (\Z_7, +)$ and $g=3$. Use problem \ref{inv_hom} to prove it for all groups.)



\section{$\phi(N)$}

Find a formula for finding $\phi(N)$ for any $N$ given its prime factorization.

Calculate $\phi(N)$ for $N = 1$ to $30$ and check that it works.



\section{Groups}

For each of these, make sure you can justify every step using definition of a group. As a refresher, here is the definition of a group $G$ (often called the ``group axioms''):

\begin{itemize}
\item Identity: There is an ``identity'' $e \in G$ such that $e \star x = x$ for every $x \in G$.
\item Inverses: For every $x \in G$, there is a $y \in G$ such that $x \star y = e$.
\item Associativity: If $x, y, z \in G$ then $(x \star y) \star z = x \star (y \star z)$
\item Closed: If $ \in G$ then $x \star y \in G$.
\end{itemize}

If $x \star y = y \star x$ for every $x, y \in G$, then $G$ is \emph{commutative}. Not every group is commutative, so \emph{you can't switch the order of elements around $\star$}.

\subsection{}

Suppose $a, b, x \in S$ are elements of any group $G = (S, \star)$. Prove that

$$
a \* x = b \* x ~~~~ \imp ~~~~ a = b
$$


\subsection{}

Suppose $G$ is \emph{not} a commutative group: $a \star b$ may not be the same as $b \star a$ for every $a, b \in G$.

Try to prove the following:

\begin{itemize}
\item $e$ is unique (Suppose there are two different identities. Show that they're the same.)
\item $e \star a = a \imp a \star e = a$. (i.e. a ``left identity'' is also a ``right identity'')
\item Suppose $x \in G$. Prove that $x \star y = e \imp y \star x = e$ (i.e. a ``right inverse'' is also a ``left inverse'').
\end{itemize}



For our course, we will usually be talking about commutative groups. If you ever take a group theory class, you will also see non-commutative groups.

%\section{$D_4$}



\section{Generators}

\begin{itemize}
\item How many elements are there in $(\Z_{1001}^*, \*)?$
\item How many generators are there in $(\Z_{1001}^*, \*)?$
\item Find a generator for $(\Z_n, +)$ for each $n$ from $1$ to as high as you can (at least $n=20$)
\item Find all the generators $\mod 23$. (First find one. Use it to find all the rest.)
\item Can you find a generator for $(\Z_{24}, +)$?
\item Pick two primes around $20$ to $40$. Can you find a number that is a generator mod both primes?
\end{itemize}


\section{Using $DLP$ to break $CDH$}

The Discrete Log Problem (DLP) is at least as hard as the Computational Diffie-Hellman Problem (CDH). We'll show this by demonstrating that your ability so solve the DLP allows you to solve CDH.

Let's work with $G = (\Z_{19}^*, \*)$ and the generator $g = 14$.

Finish filling in the list of powers of $g$ below. You can use it to solve the DLP (for base $g$ in the group $G$): by looking up a value in the bottom row, you can find the discrete log in the top row.

\begin{center}
\begin{tabular}{|c|c|c|c|c|c|c|c|c|c|c|c|c|c|c|c|c|c|c|}
\hline
$x$ & $0$ & $1$ & $2$ & $3$ & $4$ & $5$ & $ 6$ & $7$ & $8$ & $9$ & $10$ & $11$ & $12$ & $13$ & $14$ & $15$ & $16$ & $17$\\
\hline
$g^x$ & $1$ & $14$ & $6$ & & & & & & & & & & & & & & &\\
\hline
\end{tabular}
\end{center}

(Why doesn't the table have a column for $x = 18$?)\\\\

Calculate the following:

\begin{itemize}
\item $DLOG_{14}(4) =$
\item $DLOG_{14}(16) =$
\item $DLOG_{14}(7) =$
\end{itemize}

Solve the Computational Diffie-Hellman (CDH) problem for these values:

\begin{itemize}
\item $(g, g^a, g^b) = (14, 8, 17)$ $\imp$ $g^{ab} =$
\item $(g, g^a, g^b) = (14, 10, 8)$ $\imp$ $g^{ab} =$
\item $(g, g^a, g^b) = (17, 2, 16)$ $\imp$ $g^{ab} =$
\item $(g, g^a, g^b) = (13, 12, 8)$ $\imp$ $g^{ab} =$
\end{itemize}


Since you also found the values of $a$ and $b$, check that $(g^a)^b = (g^b)^a = g^{ab}$ for your answers.


\begin{itemize}
\item $137$
\item $143$
\item $56191$
\item $890843293$
\item $561$
\item $1105$.
\end{itemize}

In this class, we will learn methods for testing whether a number is prime or not without factoring the number. As has been suggested by this problem, the above test is not a reliable one.

\section{FLT Primality Test}


In class today we learned Fermat's Little Theorem, which says that if $p$ is prime and $a$ any integer, then
$$a^{p-1} = 1 \mod p.$$
We can sometimes use this to tell that a number isn't prime. For example, $2^{754128} = 64667 \mod 754129$, so we know that $754129$ is not prime, without even factoring $754129 = 19^2 \times 2089$ (doing the latter is a tougher computation). We say that $g$ is a `witness' for $m$ if $g^{m-1} \neq 1 \mod m$ - telling us that $m$ is not prime. So in the example, $2$ is a witness for $m$. For the following numbers, determine whether the number is prime (you can do this by hand, but I would recommend using a computer - you can, for instance, ask your calculator or WolframAlpha 'factor $x$' if you want to factor $x$). If the number is not prime, try to find a witness for that number. You may again use a computer for this computation. If you don't think there are any witnesses for a given composite number, it is alright to stop.

\begin{itemize}
\item 137
\item 143
\item 56191
\item 890843293
\item 561
\item 1105
\end{itemize}


\section{RSA}

Run the entire RSA algorithm using a friend:

\begin{itemize}
\item Select primes $p$ and $q$ and compute $N = p \* q$
\item Pick an exponent $e$ and find the inverse exponent $d$. (Find the inverse of $e$ mod $ \varphi(N)$. Recall that $\varphi(N) = (p-1) \* (q-1)$ and that you can use the Extended Euclidean Algorithm to find inverses).
\item Ask your friend to select a message $m$ and send you $c = m^e$.
\item Compute $c^d$ to decrypt the message.
\end{itemize}


\section{Extra RSA Problems}

Ask Lucas for extra RSA problems. ;-)


\section{Montgomery's Ladder}

The ``obvious'' way to calculate $g^y \mod n$ gives away some information about the number we are exponentiating. Can you characterize what information is given away?

Suppose the binary representation of $x$ has $x_i$ at the $2^i$'s place. Try the following for yesterday's homework problems:


\begin{itemize}
\item $a = g$, $b=x^2$
\item For $i = k-2, k-3, ..., 0$:
\begin{itemize}
\item If $x_i = 0$
\begin{itemize}
\item $b = a \* b$
\item $a = a^2$
\end{itemize}
\item If $x_i = 1$:
\begin{itemize}
\item $a = a \* b$
\item $b = b^2$
\end{itemize}
\end{itemize}
\item Output the final value of $a$.
\end{itemize}

Can you explain why it works?

\end{document}
