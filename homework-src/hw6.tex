
\documentclass[12pt]{article}


\usepackage{amsmath}
\usepackage{amssymb}
\usepackage{attrib}
\usepackage{color}
\usepackage{fancyhdr}
\usepackage{graphicx}
\usepackage{hyperref}
\usepackage{makeidx}
\usepackage{stmaryrd}
\usepackage{xcolor}

\usepackage{preamble/lgarron-1.4.1}


\hypersetup{%
  colorlinks=true,% hyperlinks will be coloured
  linkcolor=blue,% hyperlink text will be green
  linkbordercolor=blue,% hyperlink border will be red
}

\def\rf{{\ \overset{R}\longleftarrow\ }}

\makeindex

\usepackage{multirow}

\usepackage{fancyhdr}
\pagestyle{fancy}
\fancyhead[C]{SPCS Cryptography -- Homework 6}
\rhead{}

\lhead{}



%%%%%%%%%%%%%%%%%%%%%%%%%%%%%%%%%%%%%%%%%%%%%%%%%%%%%%%%%%%%%%%%
  
\begin{document} 

\section{Birthday Paradox}

\subsection{}


A deck of $52$ cards is shuffled and the top eight cards are turned over.

\begin{itemize}
\item What is the probability that the king of hearts is visible?
\item A second deck is shuffled and its top eight cards are turned over. What is the
probability that any visible card from the first deck matches a visible card from the second deck? 
\end{itemize}

\sol{
Question is from ``An Introduction to Mathematical Cryptography''
}


\subsection{}

Suppose there are $k$ people in a room, and assume that there are 365 possible birthdays, equally likely and independent for each person.

Use the approximation ${k(k-1) \over 2n}$ the chance that at least two of them share a birthday for the following values of $k$:

\begin{itemize}
\item $k=5$
\item $k=10$
\item $k=15$
\item $k=23$
\item $k=40$
\item $k=100$
\end{itemize}

Now calculate the exact probabilities (using a calculator / WolframAlpha). When does the estimate become unreliable?

\sol{
$$
\left(
\begin{array}{ccc}
 2 & 0.00273973 & 0.00273973 \\
 5 & 0.0271356 & 0.0273973 \\
 10 & 0.116948 & 0.123288 \\
 15 & 0.252901 & 0.287671 \\
 23 & 0.507297 & 0.693151 \\
 40 & 0.891232 & 2.13699 \\
 100 & 1. & 13.5616 \\
 360 & 1. & 177.041 \\
\end{array}
\right)
$$
}

\subsection{}

Suppose you are using $64$-bit random numbers. Roughly how many numbers can you generate before there is likely to be a collision?

What if you generate $128$-bit numbers instead? $32$-bit numbers?

\sol{$2^{32}, 2^{64}, 2^{16}$}

\subsection{}

Suppose you are generating $k$ keys, and need them all to be different. Roughly how large does the key space need to be for the following values of $k$?

\begin{itemize}
\item $k = 100$
\item $k = 10$
\item $k = 2$
\end{itemize}

\sol{
$n > k^2 / {2 p}$
}

\subsection{}

Suppose everyone uses RSA with two primes $p$ and $q$, each selected uniformly at random from all the primes with $256$ bits.

Assuming you are only worried about collisions, about how many keys can be generated so that the probability that none of the keys share any primes is $< 1 / 2^{40}$ (one in a trillion)?

You'll need to use the fact that there are about ${n \over 2 \log_e n}$ $n$-bit primes. Keep in mind that every RSA key contains \emph{2} primes.


\sol{
Solve
$$
{(2k)^2 \over {2 \* 2^{256} / 2 \log_e 2^{256}}} = 1 / 2^{40}
$$
for $k$:
$$
k \approx 2.43 \* 10^{31}
$$
(Note the answer on the previous line is in base $10$.
}


\section{Baby-Step Giant Step}

Recall the baby-step giant-step algorithm for finding $x$ given $(g, h = g^x)$

\begin{itemize}
\item $Let m = \lceil \sqrt{n} \rceil$
\item Make a lookup table:
\begin{itemize}
\item $h \* g^{0} \to k = 0$
\item $h \* g^{1} \to k = 1$
\item ...
\item $h \* g^{m} \to k = m$
\end{itemize}
\item For $i = 0, m, 2m, ..., p$:
\begin{itemize}
\item If $g^i$ is in the table as $h \* g^k$, output $i - k$.
\end{itemize}

\end{itemize}


Solve the following discrete log problem instances.

\begin{itemize}
\item $G = \Z_{29}^*$, $g=3$, $g^x = 7$ \sol{8}
\item $G = \Z_{67}^*$, $g=2$, $g^x = 53$ \sol{21}
\item $G = \Z_{257}^*$, $g=27$, $g^x = 57$ \sol{42}
\end{itemize}

How many steps did each one take you? How long would it have taken the naive way?

\subsection{}

Suppose $g \in \Z_p^*$. If $x$ is a random exponent, how many step on average will it take to solve the discrete log problem using baby-step giant-step.

\sol{
$1.5 \* \sqrt{p}$
}

\section{Hash Tables}

Ask Lucas about how to make baby-step giant-step actually run fast.

\section{Malleability}

Use the homomorphic properties of these systems to create and modify ciphertexts/signatures:

\begin{itemize}
\item RSA: $E(m) \to E(2m)$
\item ElGamal Encryption: $E(m) \to E(2m)$
\end{itemize}

Can you create any  forgeries for ElGamal Signatures?

\section{ElGamal Signatures}

Verify the following ElGamal signatures:

\begin{itemize}
\item $(p = 13, g = 2, h=8)$,  $m = 24, s = 6$
\item $(p = 29, g = 2, h=11)$, $m = 24, s = 2$
\item $(p = 137, g = 3, h=37)$, $m = 100, s = 42$
\end{itemize}

\subsection{}

Pick a partner; ask them to send you a signature. Verify it.



%
%\section{Coin Toss Game}
%
%Let's consider what it means for particular adversary to have a certain advantage, by considering a simple coin toss game where the adversary must distinguish a biased coin from a real one. In effect, we are computing how ``defective'' the biased coin is compared to a fair random number generator by how easily we can tell it apart.
%
%We have two coins, a fair coin (probability $1/2$ of HEADS, otherwise TAILS), and a coin biased towards heads (probability $p$ of HEADS, otherwise TAILS, where $p>1/2$). You may assume that the challenger and the adversary know $p$ ahead of time. To start the game, the challenger tosses a coin, as follows:
%
%\begin{itemize}
%\item In Experiment 0: The challenger tosses the fair coin.
%\item In Experiment 1: The challenger tosses the biased coin.
%\end{itemize}
%
%In both experiments, the challenger then sends the result of the toss to the adversary.
%
%\begin{enumerate}
%\item In the last step of each experiment, what does the adversary output, and why?
%\item Let $\mathcal{A}$ be an adversary that outputs $\mathcal{A}(b)$ in Experiment $b \in \{0, 1\}$. Define the advantage of $\mathcal{A}$ (similar to the advantage for semantic security), and explain what it means.
%\item Show the computation of the advantage of each of the following adversaries:
%\begin{itemize}
%\item $\mathcal{A}_1$: Always output $1$.
%\item $\mathcal{A}_2$: Ignore the result reported by the challenger, and randomly output $0$ or $1$ with even probability.
%\item $\mathcal{A}_3$: Output $1$ if HEADS was received, else output $0$.
%\item $\mathcal{A}_4$: Output $0$ if HEADS was received, else output $1$.
%\item $\mathcal{A}_5$: If HEADS was received, output $1$. If TAILS was received, randomly output $0$ or $1$ with even probability.
%\end{itemize}
%\item What is the best possible advantage in this game? (Give a proof.)
%\end{enumerate}
%
%
%\sol{
%\begin{enumerate}
%\item The adversary outputs either $0$ or $1$. (S)he does this because (s)he is trying to determine which experiment (s)he is in. Being in Experiment $0$/$1$ corresponds to having a fair/biased coin, so the output is effectively a guess for which coin was tossed.
%\item The advantage is computes how good an adversary's guess is that the coin is biased. It's the difference between how often the adversary \emph{incorrectly} guesses that the coin is biased and how often (s)he \emph{correctly} tells that it's biased:
%
%$$Adv(\mathcal{A}) = \Big|Pr[\mathcal{A}(0) = 1] - Pr[\mathcal{A}(1) = 1]\Big|$$
%\item
%\begin{itemize}
%\item $Adv(\mathcal{A}_1) = |1 - 1| = 0$
%\item $Adv(\mathcal{A}_2) = |1/2 - 1/2| = 0$
%\item $Adv(\mathcal{A}_3) = |1/2 - p| = p-1/2$
%\item $Adv(\mathcal{A}_4) = |p - 1/2| = p-1/2$
%\item $Adv(\mathcal{A}_4) = |3/4 - (p + {1\over 2}(1-p))| = p/2-1/4$
%\end{itemize}
%
%Note that any adversary that ignores the information from the coin toss has advantage $0$.
%
%\item $\mathcal{A}_3$ and $\mathcal{A}_4$ have the optimal advantage, $p - 1/2$.
%
%Let $X \in \{H, T\}$ be the outcome of the toss reported by the challenger to the adversary (either HEADS or TAILS). The adversary has no information other than $X$ to distinguish the experiments, so we can define:
%$$f(x) = Pr\Large[\mathcal{A}(\cdot) = 1~|~X=x\Large]$$
%
%$f$ is effectively the probability our strategy guesses ``biased'' (Experiment $1$) in each case. We can split the probability into cases based on $X$:
%$$Adv(\mathcal{A}) = \Big|
%\Big({1\over2}f(H) + {1\over2}f(T)\Big)
%-
%\Big(p\cdot f(H) + (1-p)\cdot f(T)\Big)
%\Big|$$
%$$= \Big|
%\Big({1\over2} - (1-p)\Big) f(T)
%-
%\Big(p - {1\over2}\Big) f(H)
%\Big|$$
%$$= \Big(p - {1\over2}\Big) \cdot \Big|
%f(T) - f(H)
%\Big|$$
%The best strategy is to make sure the output is as different as possible depending on $X$. We could select $f(T) = 0$ and $f(H) = 1$, which corresponds to the strategy of adversary $\mathcal{A}_3$. (The opposite choice gives us $\mathcal{A}_4$).
%\end{enumerate}
%}
%



\end{document}
